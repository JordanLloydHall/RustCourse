

\documentclass{report}
\title{%
    Rust Lab - Register Machines and Godel Numberings \\
    \large Intro to Rust Course
    }
\author{Jordan Hall - jh4020@ic.ac.uk}
\date{21/12/21}
\usepackage{mathtools}
\usepackage{listings}
\usepackage{amsfonts}
\usepackage{hyperref}
\usepackage{graphicx}
\graphicspath{ {./images/} }
\newcommand{\centerimg}[2]{\begin{center}\includegraphics[#1]{#2}\end{center}}


\DeclarePairedDelimiter\floor{\lfloor}{\rfloor}
\newcommand*{\defeq}{\stackrel{\text{def}}{=}}
\setlength\parindent{0pt}
\begin{document}
\maketitle

\section*{Summary}
Register Machines are covered in the COMP50003 Models of Computation course. 
They are simple machines that only make use of 3 unique instructions, yet are 
Turing Complete. In this task, we will deal with programming several tools that 
automate common tasks.

\section*{Before you get started!}
It's worth mentioning that you should be working on a fork of the original repository.
That way, when you're done you can submit a pull request on Github and the markers can
give feedback directly on the pull request.

\section*{Register Machines}
Register Machines are an abstract class of machines. They are Turing Equivalent 
and as a result, are useful proxies for proving statements about computability.\\
The Register Machines in the course have 3 unique instructions:
\begin{itemize}
    \item {\makebox[2.2cm][l]{$ R^+_n \to L_m $} Increment register $R_n$ by one and jump to 
    instruction $L_m$}
    \item {\makebox[2.2cm][l]{$ R^-_n \to L_m,L_p $} If register $R_n>0$ then decrement it by one 
    and jump to instruction $L_m$ else jump to instruction $L_p$}
    \item {\makebox[2.2cm][l]{$\text{HALT}$} Stop the execution of the machine}
\end{itemize}
It’s worth noting that if a jump to an instruction that is undefined occurs, 
then it’s considered a signal to halt the machine’s execution.

\section*{Godel Numbering}
Godel Numbering can be used to create bijections between Register Machine programs 
and natural numbers.

Namely, we use two different pair encodings:
pub
$ \langle\langle x,y \rangle\rangle \defeq 2^x(2y+1) $ which is a bijection between $\mathbb{N} \times \mathbb{N}$ and $ \mathbb{N^+}$

$ \langle x,y \rangle \defeq 2^x(2y+1) - 1 $ which is a bijection between $\mathbb{N} \times \mathbb{N}$ and $ \mathbb{N}$\\

It is worth noting that $2^x(2y+1)$ has $x$ least-significant bits as zeros. 
This makes it easier to decode than other Godel Numbering systems that must 
make use of prime factorization.\\

We can then define a bijection between lists of naturals and natural numbers:
$ \ulcorner [] \urcorner \defeq 0$

$ \ulcorner x::l \urcorner \defeq \langle\langle x,\ulcorner l \urcorner \rangle\rangle$\\

And we can define a bijection between Register Machine program instructions and 
natural numbers:

$ \ulcorner R^+_i \to L_j \urcorner \defeq \langle\langle 2i,j \rangle\rangle$

$ \ulcorner R^-_i \to L_j,L_k \urcorner \defeq \langle\langle 2i+1,\langle j,k \rangle \rangle\rangle$

$ \ulcorner \text{HALT} \urcorner \defeq 0 $

Combining the three numberings gives a method for representing any Register Machine
program as a natural number.

\section*{What To Do}
For this lab we want to automate the functionality described above to create a
suite of useful tools.

\subsection*{Program Evaluation}
Being able to evaluate programs on a certain set of inputs would be very useful
and so we define representations of machine state and programs as such:
\centerimg{width=\textwidth}{ProgramTypes}
We rename primitive types such as usize and u128 in order to make them more readable.
Note that this is just type aliasing. \\


As for state, or configuration, we use a
tuple of the current label of the instruction being executed and a hashmap that
maps registers to their value. We choose to use a hashmap as there's no guaranteed
pattern of register numbering in a program (e.g. a program may use registers 1, 100, 100000, etc).\\


The Instruction enum is used to encode a choice between three unique instruction types.
A wonderful thing about Rust is that enums can hold data much like a struct, and
this data need not be the same for each enumeration. For Add, we store the Register we
want to increment, and then the Label of the Instruction we jump to next. For Sub,
we store the Register we want to decrement, and then the Label of the Instruction we 
jump to if that Register wasn't zero followed by the Label of the Instruction we want to
jump to if that Register is zero.\\


Now we define a function that, given a list of Instructions and an initial State,
returns State of the program when it halts (note that this evaluator will run
indefinitely if given a non-terminating function).\\


As an aid, we give the function signature:
\centerimg{width=\textwidth}{EvalProgram}

\subsection*{Godel Numbering}
Converting between the different representations of lists, programs, and pairs can
be rather tedious. In this part, we will write some functions that convert between
the representations in an easy-to-use way.


Firstly, encoding pairs of natural numbers their Godel Numbering representations
is a building block that will be used to create the other converters.
\\
Note: $\text{encode\_pair1}(x,y) \defeq \langle\langle x,y \rangle\rangle $ and 
$\text{encode\_pair2}(x,y) \defeq \langle x,y \rangle $\\

As an aid, we give the function signatures:
\centerimg{width=\textwidth}{EncodePairs}

The next step is to encode a list of natural numbers into its Godel Numbering representation:
\centerimg{width=\textwidth}{EncodeList}

And finally, encode every Instruction in a program into it's Godel Numbering representation:
\centerimg{width=\textwidth}{EncodeProgram}

With all of these functions, a program can be converted into a natural number.
This is useful if you want to instruct a Universal Register Machine what Register 
Machine program to run.\\

However, we also would like to convert between these representations in
reverse as well. If not, then what is the whole point in having these bijections!?:
\centerimg{width=\textwidth}{Decodes}

\section*{Testing}
For this lab, we have made some simple unit tests in order to show how unit tests
are formatted, as well as test small elements of your code's functionality.
You may run these tests by opening a terminal in this lab's root directory and typing:
\begin{lstlisting}[language=Bash]
$ cargo test
\end{lstlisting}

The tests given for this lab are quite sparse and do not tests edge cases. We highly
encourage you to extend the test suite to account for any edge/corner cases your code]
may come accross.

\section*{Extensions}

For this lab we have elected to run three extensions
\subsection*{Extension A} 
Extend the code to make use of the \href{https://docs.rs/num-bigint/latest/num_bigint}{num\_bigint} and 
\href{https://docs.rs/num-traits/latest/num_traits}{num\_traits} crates.
As of now, the maximum values that can be stored in Registers, and manipulated by
the encoding/decoding functions, is $2^{128}-1$. This is far too small and makes doing anything
\textbf{\textit{useful}} almost impossible. BigInts allow us to manipulate arbitrary-sized
integers (or unsigned integers, as is for this lab) limited only to the size of your
main memory.

An updated skeleton for this task would look like:
\centerimg{width=\textwidth}{BigUInt}
Note that BigUint does not implement the Copy trait. If we don't want to move the BigUint when
passing them as parameters to function calls, then we must borrow the BigUint by passing
the function call a reference to the BigUint. BigUints do implement the Clone trait.

\subsection*{Extension B}
The code in this state is certainly useful for someone who knows how to use Rust,
but what about the non-programmers who have just learned abour Register Machines in
their Intro to English Literature course? How are they meant to interact with your 
code?

Of course this is in jest, but it's an important idea to make your code usable by
the general public.
Use the \href{https://github.com/clap-rs/clap}{clap} crate to create an interface
with which users may use functionality in your code. Feel free to be creative here!
Do you want your users to pass in a text file that has the program? How about a way
for the user to just pass in a Godel Number for something, and encode it/
decode it into another representation? Maybe you want to allow the user to step 
through the execution of some program instruction-by-instruction. Really, go wild!

\subsection*{Extension C}
This extension is definitely meant as more of a flex. Towards the end of the
COMP50003 course, we were introduced to the idea of a Universal Register Machine.
That is a Register Machine which can take the Godel Numbering for any other Register
Machine and execute it. The link to the video is 
\href{https://imperial.cloud.panopto.eu/Panopto/Pages/Viewer.aspx?id=d3732fc3-0d12-40be-9e41-aded00929300}{here}.
Try to write such a program that can be executed with your evaluator. See what interesting
things you can come up with. What happens if you encode the Universal Register Machine
into its Godel Numbering, and pass that into the Universal Register Machine?

\section*{Submission}
Once you've finished working and you want the piece to be marked, you can create 
a pull request on Github, and a marker will come along and give you interactive
feedback on the pull request. This means that marking can be a two-way conversation
between you and the marker.

Before you start a pull request, it is worth rebasing your forked repository with
the original:
\begin{lstlisting}[language=Bash]
$ git remote add upstream https://github.com/JordanLloydHall/RustCourse
$ git rebase -i upstream/main
\end{lstlisting}

This can also be used to ensure that you have the lastest spec for the course as well! 
We are no strangers to mistakes and this allows us to fix them easily.

\end{document}